% !TeX spellcheck = ca
\documentclass{article}
\usepackage[utf8]{inputenc}
\usepackage{graphicx}
\usepackage{tikz}
\usepackage{listings}
\usepackage{float}
\usepackage{amsmath}
\usetikzlibrary{positioning,fit,calc,arrows.meta, shapes}
\graphicspath{ {images/} }

%Tot això hauria d'anar en un pkg, però no sé com és fa

\newcommand*{\assignatura}[1]{\gdef\1assignatura{#1}}
\newcommand*{\grup}[1]{\gdef\3grup{#1}}
\newcommand*{\professorat}[1]{\gdef\4professorat{#1}}
\renewcommand{\title}[1]{\gdef\5title{#1}}
\renewcommand{\author}[1]{\gdef\6author{#1}}
\renewcommand{\date}[1]{\gdef\7date{#1}}
\renewcommand{\contentsname}{Índex}
\renewcommand{\maketitle}{ %fa el maketitle de nou
	\begin{titlepage}
		\raggedright{UNIVERSITAT DE LLEIDA \\
			Escola Politècnica Superior \\
			Grau en Enginyeria Informàtica\\
			\1assignatura\\}
		\vspace{5cm}
		\centering\huge{\5title \\}
		\vspace{3cm}
		\large{\6author} \\
		\normalsize{\3grup}
		\vfill
		Professorat : \4professorat \\
		Data : \7date
\end{titlepage}}
%Emplenar a partir d'aquí per a fer el títol : no se com es fa el package
%S'han de renombrar totes, inclús date, si un camp es deixa en blanc no apareix

\renewcommand{\figurename}{Figura}
\renewcommand{\tablename}{Taula}
\title{Práctica 2}
\author{Joaquim Picó Mora, Ian Palacín Aliana}
\date{3 de Desembre 2019}
\assignatura{Sistemes Concurrents i Paral·lels}
\professorat{F. Cores}
\grup{PraLab1}

%Comença el document
\begin{document}
	\maketitle
	\thispagestyle{empty}
	
	\newpage
	\pagenumbering{roman}
	\tableofcontents
	\newpage
	\pagenumbering{arabic}
	


% Petita estimació de si el sistema serà útil per l’empresa i és factible de ser desenvolupat 
% sota les restriccions existents, a fi de determinar si es tira endavant, o bé val la pena invertir 
% en un estudi de viabilitat més profund i seriós
%TODO: @horno Fes aquesta part

%. Domini  (NO es demana)
%Glossari de termes del domini

\section{Introducción}

En este documento se compara la eficiencia en tiempo que supone ejecutar Indexing y 
Query de forma concurrente respecto de forma secuencial. 
También se muestra la diferencia en tiempo que 
supone la ejecución del programa con un número distinto de hilo.
\\\\
Los datos que se han utilizado para realizar el estudio salen de la ejecución 
de forma secuencial y concurrente de múltiples ejemplos, calculando así 
el tiempo que tarda en realizarse cada una de las tareas.
 Este tiempo será el que se usará para sacar conclusiones.\\
Del mismo modo, se ejecutará varias veces el programa con distinto número
 de hilos, todos estos datos quedan recogidos en las gráficas.

\section{Concurrente vs Secuencial}

\begin{figure}[hbt!]
  \includegraphics[width=\linewidth]{Indexing.png}
  \caption{Indexing}
  \label{fig:indexing}
\end{figure}
\begin{figure}[hbt!]
  \includegraphics[width=\linewidth]{Query.png}
  \caption{Query}
  \label{fig:query}
\end{figure}
\newpage
%escriure a partir d'aquí
El objetivo de la práctica ha sido implementar la versión concurrente de 
Query y Indexing, así como estudiar su comportamiento y tiempos de ejecución.
\\\\ 
En las gráficas se pueden discernir los tiempos de ejecución
de cada uno de los ejemplos en su versión secuencial y concurrente,
 sin embargo, cabe destacar algunas peculiaridades sobre los resultados obtenidos.\\
\\
El resultado más chocante es que en algunos casos la ejecución 
concurrente da un tiempo superior a la implementación secuencial,
esto se debe lo siguiente:
La implementación tanto en Query como en Indexing se basa en una
estrategia Master/Worker que consiste en dividir la construcción del
HashMap de Inverted Index y repartirlo en diferentes hilos, haciendo que cada
hilo construya su propio HashMap parcial. Posteriormente se unen los HashMaps
de cada uno de los hilos para generar el acumulado. Esta última
 operación (putAll) genera un cuello de botella que se suma al de I/O haciéndola bastante costosa
en tiempo, motivo por el que se explica que a veces la aplicación concurrente
sea menos eficiente que la secuencial. Estamos convencidos que hay implementaciones
que pueden rodear o prescindir de putAll, en ese caso la implementación concurrente
no solo no sería más lenta que la secuencial, sinó que sería más rápida. Una posible 
solucion sería sustituir HashMultiMap por algun HashMap thread safe e ir 
acumulando las claves sobre el mismo objeto.

\section{Diferencias entre número de hilos}
\begin{figure}[hbt!]
  \includegraphics[width=\linewidth]{multithreading.png}
  \caption{Ejecucion con múltiples hilos}
  \label{fig:multhilos}
\end{figure}

Como se puede ver en la Figura \ref{fig:multhilos}, en el caso tanto de Query
 como de Indexing conforme mas hilos ejecutándose de forma paralela, más coste
  en el tiempo.\\
Esto es debido a lo ya mencionado en el apartado anterior, a más hilos más veces
se tiene que realizar la operación putAll, alargando más el cuello de botella. Se 
le tiene que añadir, también, el coste de creación y mantenimiento de más hilos.\\

\section{Diseño de la Solución}
La solución presentada en esta práctica consiste en lo siguiente.
\subsection{Indexing}
Se recibe por parámetro el número de hilos con los que se desea ejecutar la
aplicación. Se realiza un balanceo de carga asignando a cada hilo un
número determinado de caracteres a procesar del fichero. Una vez asignado a cada
hilo la carga de trabajo a realizar, estos empiezan a procesar carácter a carácter
las claves y a construir sus HashMap parciales. Una vez realizado el join de todos
los hilos, se juntan los HashMaps parciales en uno y este se guarda o se imprime según
los argumentos especificados.
%TODO: Si vols, explicar lo de la mierda de Save Index així una mica per demunt i lo del cuello de botella de remove
\subsection{Query}
Utiliza la misma estructura que Indexing, pero esta vez el balanceo de carga se
realiza respecto a los ficheros que deben leer cada uno de los hilos. Cada thread
leerá un número equitativo de ficheros y creará su propio hashmap así continuando
con la misma arquitectura que la concurrencia en Indexing.

\section{Apuntes finales}
La parte donde se guardan las claves de forma concurrente está comentada debido
a unos problemas, no obstante nos gustaría acentuar algunas curiosidades.
\\\\
De forma muy parecida a putAll, la mayoría de tiempo de esta función la acarriaba una 
llamada a un método, en este caso era remove() de ArrayUtils. Este método lo
utilizábamos para borrar las llaves leídas y preparar el set de llaves para
el siguiente thread. Como curiosidad, en textos muy extensos este método ocupaba
hasta un 97\% del tiempo de ejecución del guardado de los índices.
\end{document}

